\section{Executive Summary}
\iffalse
The purpose of this introduction to your idea is to clearly and succinctly describe the final goal of the project. It should list key features and components and explain why the project is interesting and worthwhile.

When "searching for funding/approval" you have very little time/space to capture the interest of the reader. You need to concisely describe what the project will do, what need it will address, and why a completed project with be of benefit. You also need to convey interest, enthusiasm, and determination. You want the reader coming away from this first page eager to know more and excited about the prospects.
 
This section should be one page long.

NOTES:
Look at: 
"A meta-analysis of the effectiveness of computer-assisted instruction in science education"

"Success in introductory college physics: the role of high school preparation"

"STEM Attrition: College Students' Paths Into and Out of STEM Fields"

"Trends in International Mathematics and Science Study"

\fi

Physics is one of the core subjects that college-bound high school students will explore, especially for those that plan on pursuing a degree in a scientific field. However, many instructors have their doubts about how well high school courses are preparing students for their college experience. According to a 2013 study by U.S. Department of Education, 46\% of beginning bachelor's degree students in the physical sciences either left their post-secondary education without a degree or changed majors. For those beginning an associates degree, 64\% of such students did the same. %Stem attrition%
% Also find statistics on dropout rates here, TIMSS could be useful
% Statistics on the benefits of using software in education?
% First study I found indicates it makes little difference...

Clearly, there is a need for improvement in this area. Our team believes that we can provide a web-based solution that would have the power to make a difference for both high school students and students taking college level physics for the first time. Our project will be called \textit{Principia}, named after \textit{Philosophiæ Naturalis Principia Mathematica}, the collection of books written by Newton that serve as the foundation of classical mechanics. 

We envision Principia as a website that is in equal parts an engaging community experience and a powerful tool to model physical systems. Students will have a support network that will help them overcome challenges and a chance to share their knowledge with others.

The core feature will be an interactive sandbox for designing systems. It will include a toolbox that contains components like point masses, springs, and pulleys. Users can drag and drop these components into a central canvas. Within the canvas, any instance of a component can be selected and its properties will be displayed on the side of the window. See section x for an enumeration of the components we will support and section y for figures demonstrating our plans for the UI. 

Once the components are in place, users can select play and the canvas will come to life. The components will move over time according to their properties, allowing students to visualize the system. At any point, they can pause the simulation and select a component to view its properties, allowing them to get quantitative results in addition to the visualization. 

Principia will feature much more than the sandbox. To enable users to connect with and support one another, the website will support the creation of accounts with additional features. A registered user can save and load their creations, share them with others, browse simulations, provide comments and annotations, and even create quizzes and walkthroughs. Imagine a future where instructors integrate Principia into their course. No longer will students work in isolation, writing out equations based on a static image or an ambiguously worded problem. Instead, students will have the chance to engage with both the instructor and their peers, exploring problems interactively or even inventing their own systems.

% Replace fantastic with a better word
We see the final version of Principia as potentially being the number one site for physics education as a fantastic supplement to any physics classroom as well as a tool that empowers motivated individuals to educate themselves with the support of a community.